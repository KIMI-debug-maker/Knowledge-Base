% English Article template created by Vopaaz
\documentclass{article}
\usepackage{geometry}
\geometry{a4paper}
\usepackage{setspace}
\usepackage{enumerate}
\usepackage{enumitem}
\usepackage{hyperref}
\hypersetup{colorlinks,allcolors=black}

\setenumerate[1]{itemsep=0pt,partopsep=2pt,parsep=0pt ,topsep=2pt}
\setitemize[1]{itemsep=0pt,partopsep=2pt,parsep=0pt ,topsep=2pt}
\setenumerate[2]{itemsep=0pt,partopsep=2pt,parsep=0pt ,topsep=2pt}
\setitemize[2]{itemsep=0pt,partopsep=2pt,parsep=0pt ,topsep=2pt}
\setdescription{itemsep=0pt,partopsep=2pt,parsep=0pt ,topsep=2pt}

\usepackage{graphicx}
\usepackage{fontspec}

\defaultfontfeatures{%
	RawFeature={%
		% +swsh,
		+calt
	}%
}

\setmainfont{EB Garamond}

\usepackage{xcolor}
\usepackage{xspace}
\usepackage{float}

\usepackage{parskip}
\usepackage{multirow}

\usepackage{amsmath}
\usepackage{amssymb}

%-----------%

\title{POM Summary}
\author{YiFan Li}
\date{\today}

\newcommand{\red}[1]{\color{red}#1\color{black}\xspace}

\renewcommand{\arraystretch}{1.25}

\begin{document}

\addfontfeatures{RawFeature={+smcp}}
\maketitle
\addfontfeatures{RawFeature={-smcp}}

\tableofcontents
\clearpage

%-------%

\section{Introduction}

What is Operations?
Any activity that takes one or more \red{inputs}, \red{transforms} them, and provides one or more \red{outputs} for its customers.

What is OM?
The systematic \red{design}, \red{planning} and \red{control} of processes that transform inputs into services and products.

Processes can be linked to form a supply chain. Supply chain management is the synchronization of a firm's process with its suppliers and customers.

Goal of OM:
\begin{itemize}
	\item Quality
	\item Time
	\item Cost
	\item Flexibility: diversity of products' volume, color, model, etc.
\end{itemize}

Three key elements for business:
\begin{itemize}
	\item Strategy: Business goal
	\item Operations: How to rightly achieve the business goal
	\item Leadership: Motivate and guide the employees
\end{itemize}


\section{Operations Strategy}

Three aspects:

\begin{itemize}
	\item How to source? -- Supply Chain design
	\item How to choose production mode? -- Production and Inventory Strategies
	\item How to highlight Product's features? -- Competitive Priorities and Trade-offs
\end{itemize}

\subsection{Supply Chain Design}

A supply chain is the interrelated series of processes,
within a company and across different companies, that produces a service or product.

The basis of supply chain design is a make-or-buy decision about which of the processes should a company own or outsource.

The answer to above question determines the extent of a company's \textbf{vertical integration}.
Making more means higher vertical integration.

How to decide? Break-even analysis.

\begin{figure}[H]
	\centering
	\includegraphics[width=0.6\linewidth]{img/break-event-analysis.png}
\end{figure}

Other considerations for make-or-buy decision:

\begin{itemize}
	\item Comparative labor costs
	\item Logistics and communication costs
	\item others
\end{itemize}

How to make change? Backward and Forward Vertical Integration.

\begin{description}
	\item[Backward Integration] A company acquires its suppliers
	\item[Forward Integration] A company acquires its distribution channels
\end{description}

\subsection{Competitive Priorities and Trade-offs}

Four groups of competitive priorities:

\begin{table}[H]
	\centering
	\begin{tabular}{|c|c|l|c|}
		\hline
		Field                        & Aspect             & \multicolumn{1}{c|}{Definition}            & Example    \\ \hline
		Cost                         & Low-cost           &                                            & Wal-Mart   \\ \hline
		\multirow{2}{*}{Quality}     & Top quality        &                                            & Apple      \\ \cline{2-4}
		                             & Consistent quality & Same quality in different time and regions & McDonald's \\ \hline
		\multirow{3}{*}{Time}        & Delivery speed     & Quick                                      & Dell       \\ \cline{2-4}
		                             & On-time delivery   & Meet promises                              & Jing Dong  \\ \cline{2-4}
		                             & Development speed  & Quickly introducing new service or product & Zara       \\ \hline
		\multirow{3}{*}{Flexibility} & Customization      &                                            & Cute smart \\ \cline{2-4}
		                             & Variety            &                                            & Lenovo     \\ \cline{2-4}
		                             & Volume flexibility & \begin{tabular}[c]{@{}l@{}}Accelerating or decelerating the rate\\ of production quickly\\ according to the demand fluctuations\end{tabular}                  &            \\ \hline
	\end{tabular}
\end{table}

It's not possible to meet them all simultaneously. Companies need to make trade-offs.

For a company in the short-term, higher variety means higher cost.
However, in the long run (or when comparing two companies), the variety and cost can be both improved.

\begin{figure}[H]
	\centering
	\includegraphics[width=0.7\linewidth]{img/cost-flexibility-trade-off.png}
\end{figure}

Each line represents a company. The company represented by the black line is better.

\subsection{Production/Inventory Strategies for Manufacturers}

Three kind of strategies:

\begin{itemize}
	\item Make-to-Stock: Standardized products
	\item Assemble-to-Order: Semi-customized products.
	      There are fixed models of accessories, but customer can choose how to assemble them.
	\item Make-to-Order: Customized products.
	      The difficulty lies at fulfilling speed and cost at the same time.
\end{itemize}


\section{Process and Bottleneck Management}

\subsection{Product-Process Matrix}

\begin{figure}[H]
	\centering
	\includegraphics[width=0.6\linewidth]{img/ppmatrix.png}
\end{figure}

X-axis: Volume, Y-axis: Variety.

The most efficient production processes are found around the diagonal of the matrix.


\begin{itemize}
	\item Continuous process: a very Standardized product is needed in a large volume.
	      Automated, no flexibility. e.g. BP (Petroleum industry)
	\item Assembly line: consumers demand some degrees of customization, but still in a high volume.
	      Still referred as mass production. e.g. Honda (vehicle)
	\item Batch processing: volume is lower and variety is greater.
	      A disconnected production line is involved.
	      Worker comes to the machine rather than the product (inventory) come to them.
	      e.g. Caterpillar (construction machine)
	\item Job shop: highly specialized product is needed in small volume, based on each customer's order.
	      e.g. AED vision (eyeglasses)
\end{itemize}

Strategic fit:
\begin{itemize}
	\item Top quality and flexibility: use job processes or small batch process
	\item Low-cost operations and delivery speed: use large batch, line, or continuous process
\end{itemize}

\section{Process Analysis}

A set of concepts and tools used to describe, measure,
and ultimately improve process and operating system.

Concept definition:
\begin{description}
	\item[Cycle time] the average time between completion of successive batches (or units)
	\item[Throughput time] the length of time that unit takes between entering
	      and leaving the process (including any in-process storage or transport time)
	\item[Capacity] the maximum output rate from a process
	\item[Bottleneck] a work station with the longest cycle time
\end{description}

\subsection{Bottleneck Management}

\subsubsection{Identifying bottleneck of a single-product process}

\begin{itemize}
	\item a serial of process: cycle time is the maximum sub-process cycle time.
	      e.g. the purple ``60 min/batch''
	\item several parallel processes:
	      cycle time is the sub-process cycle time divided by degree of parallelism e.g. the red ``30 min/batch''
\end{itemize}

\begin{figure}[H]
	\centering
	\includegraphics[width=0.9\linewidth]{img/process-analysis-example.png}
\end{figure}

Finally, the ``Pack'' is the bottleneck.

\subsubsection{Related manangement decisions}

All non-bottleneck processes should have idle time so that their actual
cycle time is in accordance with the bottleneck.

If there are parallel processes, they should not output at the same time,
instead, they should interlace.

Steps for bottleneck management:
\begin{enumerate}
	\item Identifying bottlenecks
	\item Relieving bottlenecks
	\item Maximizing marginal benefits of bottlenecks
\end{enumerate}

\subsubsection{Identifying bottleneck of a multiple-product process}

\begin{figure}[H]
	\centering
	\includegraphics[width=0.9\linewidth]{img/multiple-product-process.png}
\end{figure}

Steps to identify the bottleneck:
\begin{enumerate}
	\item Calculate the total required time of each product at each station
	\item Calculate the total required time of a station
\end{enumerate}

\begin{figure}[H]
	\centering
	\includegraphics[width=0.9\linewidth]{img/multiple-product-process-solution.png}
\end{figure}

If the productivity of the bottleneck is a hard constraint, try to maximize its marginal benefit.
That is, produce the product which has the maximum (unit price / process time at bottleneck) first.

\section{Capacity Planning and Forecasting}

\subsection{Measures of Capacity}

Use Output Measures when process has high volume and standardized products,
e.g. automobile, PC, mobile phone

Use Input Measures when product variety and process divergence,
e.g. hospital, software development

\subsection{Capacity Utilization and Cushions}

\[
	\text{Utilization} = \dfrac{
		\text{Average output rate}
	}{
		\text{Maximum capacity}
	} \times 100\%
\]

\begin{center}
	Capacity cushion = 100\% - Average Utilization rate(\%)
\end{center}

Capacity cushions are needed to handle sudden changes.

\subsection{Capacity Planning and Management}

Capacity Planning for the long-term.
Manager \red{make capacity expansion} to invest in new equipment to meet increasing demand.

Capacity Management for the short-term.
Manager \red{utilize capacity elasticity} to meet fluctuating demand,
within the mixed equipments.

Three capacity timing and sizing strategies:
\begin{itemize}
	\item Expansionist: Plan extra unused capacity, more than the capacity forecast
	\item Wait-and-see: Plan insufficient capacity, use short-term options to fill the gap
	\item Intermediate: Compromise between the above two
\end{itemize}

Approaches to long-term capacity decisions:
\begin{enumerate}
	\item Estimate future capacity requirements
	\item Identify gaps between requirements and available capacity
	\item Develop alternative plans for reducing the gaps
	\item Evaluate each alternative, and make a final choice
\end{enumerate}

\subsubsection{Estimate Capacity Requirements}

For a process that makes a single product, the number of ``machine'' or ``workstation'' required is

\[
	M = \dfrac{D \cdot p }{N\left[1 - C\%\right]}
\]

where:
\begin{itemize}
	\item $D$: demand number (unit) forecast for the year
	\item $p$: processing time per unit
	\item $N$: total number of hours per year during which the process operates
	\item $C$: desired capacity cushion
\end{itemize}

For a process that makes multiple products:

\[
	M = \dfrac{
	\sum \limits _{\text{product}} \left[D\cdot p + (D/Q)\cdot s\right]_{\text{product}}
	}{N\left[1-C\%\right]}
\]

where:
\begin{itemize}
	\item $Q$: number of units in each batch
	\item $s$: setup time (the length of time it takes to switch from making one type of product to another) per batch.
\end{itemize}

In brief, the added item $(D/Q)\cdot s$ means the total setup hours spent on this product per year.

\subsubsection{Work Measurement Techniques}

Estimating the $p$ in the previous formula, which is the processing time per unit,
with the following work measurement techniques.

\begin{description}
	\item[Time Study Method]
	      Use a trained analyst to set a time standard in four steps:
	      \begin{enumerate}
		      \item select the work elements (steps in a flowchart) within the process to be studied
		      \item timing the elements (with a stopwatch)
		      \item determining the sample size
		      \item setting the final standard (considering break allowance, unavoidable delays, etc.)
	      \end{enumerate}
	\item[Elemental Standard Data Approach]
	      Often applied when products are highly customized. Use a database of standards compiled
	      by a firm's analysts for basic element that they can draw on later to estimate the time required
	      for a particular job.
	\item[Predetermined Data Approach]
	      Divide each work element into a series of micro-motions that make up the element.
	      Analysts consults a published database that contains the normal times for the full
	      array of possible micro-motions. The process's normal time is the sum of them.
	\item[Work Sampling Method]
	      Estimate the proportion of time spent on different activities,
	      based on observations randomized over time.
	\item[Learning Curve Analysis]
	      With instruction and repetition, workers learn to perform jobs more efficiently.
	      A Learning Curve displays the relationship between processing time and the
	      \textbf{cumulative} quantity of a product or service produced.
	\item[Process Charts]
	      Document all the activities performed using a table, and provides information about each step in the process.
	      The table requires the time estimates.
\end{description}


\subsubsection{Decision Making under Risk}

When the demand is uncertain but the probability of which is known,
a decision tree can be used to estimate the payoff.
Note that the final decision should not depend only on the payoff,
instead, the risk should also be taken into account,
which depend on the manager's risk preference.

\subsection{Demand Forecasting}

In POM, demand forecasting have two tiers.
\begin{itemize}
	\item Tier 1: forecast for product families for longer periods
	\item Tier 2: forecast for individual product (SKU) for shorter time periods
\end{itemize}

\begin{figure}[H]
	\centering
	\includegraphics[width=0.7\linewidth]{img/two-tier-prediction.png}
\end{figure}

\subsubsection{Qualitative Methods - Judgement Methods}

Judgemental forecasts use contextual knowledge gained through experience.

\begin{itemize}
	\item Salesforce estimates
	\item Executive opinion
	\item Market Research
	\item Delphi method (Professionalist opinion roll polling)
\end{itemize}

\subsubsection{Quantitative Methods - Casual Methods}

Linear regression: $Y = a + bX$, where $b = \dfrac{
		\sum XY - n\bar{X}\bar{Y}
	}{
		\sum X^2 - n\bar{X}^2
	}$, $a= \bar{Y} - b \bar{X}$

Sample correlation coefficient: $r = \dfrac{
		\sum (X - \bar{X})(Y - \bar{Y})
	}{
		\sqrt{
			\sum (X - \bar{X})^2(Y - \bar{Y})^2
		}
	}$

Standard error of the estimate: $s_{yx} =
	\sqrt{
		\dfrac{
			\sum Y^2 - a \sum Y - b \sum XY
		}{
			n - 2
		}
	}$

\subsubsection{Quantitative Methods - Time Series Methods}

Simple moving average:

\[
	F_{t+1} = \dfrac{
		D_t + D_{t+1} + \cdots + D_{t-n+1}
	}{n}
\]

where
\begin{itemize}
	\item $D_t$: actual demand in period $t$
	\item $n$: total number of periods in the average
	\item $F_{t+1}$: forecast for period $t+1$
\end{itemize}

Weighted Moving Averages:

\[
	F_{t+1} = W_1D_t + W_2D_{t+1} + \cdots + W_nD_{t-n+1}
\]

where $w$ is the weights that should be added to 1.

Sophisticated (exponential) weighted moving average.

\[
	F_{t+1} = \alpha D_t + (1-\alpha)F_t
\]

where
\begin{itemize}
	\item $F_t$: the last period's forecast for this period
	\item $D_t$: the demand for this period
	\item $\alpha$: a smoothing parameter between 0 to 1
\end{itemize}

It emphasizes the recent data and tend to ignore the data before long,
decreasing exponentially.

\subsubsection{Measures of Forecast Error}

Given $E_t = D_t - F_t$. we have the following measures of forecast error.

\begin{itemize}
	\item CFE$=\sum E_t$, cumulative sum of forecast errors
	\item MSE$=\frac{\sum E_t^2}{n}$, mean squared error
	\item MAD$=\frac{\sum |E_t|}{n}$, mean absolute deviation
	\item MAPE$=\frac{
			      (\sum |E_t| / D_t)
		      }{n}$, mean absolute percent error
\end{itemize}


\section{Location}

\subsection{Issues of Location Decision}

\begin{itemize}
	\item Two levels
	      \begin{itemize}
		      \item Select an area
		      \item Select a specific site in the area
	      \end{itemize}
	\item Two kinds
	      \begin{itemize}
		      \item Locating a single facility
		      \item Locating a facility within a supply chain network
	      \end{itemize}
\end{itemize}



\subsection{Locating a Single Facility}

\subsubsection{Breakeven Analysis}

A site will have its fixed cost caused by land, equipment and buildings.
It will also have a variable cost per unit, caused by labor, materials, transportation, etc.

Then we can conduct breakeven analysis based on the predicted demand (number of units to be produced).

\subsubsection{Load-Distance Method}

Select a location that minimizes the sum of the loads multiplied by the distance the load travels.
$ld = \sum_i l_i d_i$, where $i$ is the site, $l$ is the load in this site,
$d$ is distance from this site to the selected location.

Generally you should use Manhattan distance instead of the Euclidean distance,
because the roads are usually orthogonal.

\subsubsection{Searching the Optimum Location}

Steps:
\begin{enumerate}
	\item Locating the facility at the center of gravity of the target area:
	      $x^* = \dfrac{\sum_i l_i x_i}{\sum_i l_i}$ and $y^* = \dfrac{\sum_i l_i y_i}{\sum_i l_i}$.
	      Calculate its $ld$.
	\item Using the center of gravity as the starting solution, next evaluating locations near
	      (one mile north, south, east and west). If one of these locations have a $ld$ lower, it
	      becomes the new starting solution
	\item Continue until converge
\end{enumerate}

\subsection{Locating a Facility within a Supply Chain Network}

Two situations:
\begin{itemize}
	\item Facilities operate independently
	\item Facilities interact. Three dimensions must be solved simultaneously:
	      \begin{itemize}
		      \item Location
		      \item Capacity
		      \item Allocation
	      \end{itemize}
\end{itemize}

\subsubsection{Transportation Method}

Deciding the allocation pattern that minimizes the cost of shipping products totally.
Solve this with linear programming.

This only finds the best shipping pattern between plants and warehouses for a particular plant location arrangement.
The analyst must try a variety of location-capacity combinations to find the optimal solution.

\subsubsection{Considering Risk Pooling Effect}

Building two warehouses, each of which serve a part of market, is called decentralized location.

Building a single warehouse in between, serving all market, is called centralized location.

The variability can be measured by $\text{CV} = \dfrac{\sigma}{\mu}$,
where $\sigma$ is the standard deviation and $\mu$ is the mean if the historical demand of the products.

The variability of the central warehouse will be smaller than the sum of variabilities of the
two decentralized warehouses, because of the risk pooling effect.
This allows the company to reduce safety stock and therefore reduce average inventory.

If the demand of the two markets are strongly positively correlated,
the risk pooling effect tend not to work.

\subsection{Merging All Factors - Preference Matrix}

List several criteria, each is assigned a weight representing its importance.
The total weights should be equal to 100.

Each criteria will have a score for each location option.
The total score of each option is the sum of the weighted scores.

This serves as a conclusive result of all previous methods.

\section{Inventory Management}

\subsection{Definitions}

\begin{itemize}
	\item Inventory: A stock of materials used to satisfy customer demand or to support the production of services or good
	\item Inventory Management: Management of inventories to meet competitive priorities
	\item Inventory Management System: The set of policies and controls that monitors levels of inventory and determines:
	      \begin{itemize}
		      \item What levels should be maintained
		      \item When stock should be replenished
		      \item How large orders should be
	      \end{itemize}
\end{itemize}

\subsection{Inventory Decisions}

Incentives for small inventories, holding a small number of inventories will improve them:
\begin{itemize}
	\item Cost of capital: the opportunity cost of investing in an asset relative to the expected return on assets of similar risk
	\item Storage and handling costs: incurred when a firm rents space, or when it could use the storage space for other productive purposes
	\item Taxes, Insurance
	\item Shrinkage: when inventory's values decreases over time
\end{itemize}

Incentives for large inventories, holding a large number of inventories will improve them:
\begin{itemize}
	\item Customer service: more inventories can prevent stockout (an order that cannot be satisfied) or backorder (a delayed order)
	\item Ordering cost: Cost incurred each time you place an order to the suppliers, regardless of the size of the order, e.g. paperwork
	\item Setup cost: The cost involved in changing over a machine or workspace to produce a different item
	\item Labor and equipment utilization
	\item Transportation cost
	\item Payment to suppliers: a large order is likely to have less unit cost
\end{itemize}

Or more generally, large inventories helps with:
\begin{itemize}
	\item taking advantage of economies of scale
	\item protecting against uncertainty
	\item supporting a strategic plan
\end{itemize}

\subsection{Types of Inventory}

\subsubsection{Accounting Purpose}
\begin{itemize}
	\item Raw Materials: Vendor-supplied items
	\item Work-in-Process (WIP): Items that have been partially processes but are still incomplete
	\item Finished Goods: Completed products that are still in the possession of the firm (not sold)
\end{itemize}

\subsubsection{Management Purpose}
\begin{itemize}
	\item Cycle inventory: the portion of inventory that varies directly with lot size.
	      Right after a new lot (raw material order) with size $Q$ arrives, the cycle inventory number is $Q$.
	      Right before it arrives, its number is $0$.
	\item Safety stock inventory: surplus inventory that protects against uncertainties in demand, lead time, and supply changes.
	      In the sense of statistical expectation, it should always hold constant.
	\item Anticipation inventory: inventory used to absorb uneven rates of demand and supply, which are seasonal and predictable.
	\item Pipeline inventory: inventory that is issued but not yet received.
	      Let $\bar{d}$ be the average demand within a certain period,
	      and $L$ be the item's lead time\footnote{The time between starting and completing a production process},
	      the pipeline inventory is $\bar{D}_L = \bar{d}L$
\end{itemize}

\subsection{Inventory Reduction Tactics}

\begin{itemize}
	\item Cycle inventory: reduce the lot sizes (order size)
	\item Safety stock: place orders close to time needed
	\item Seasonal inventory: Match demand rate with production rate
	\item Pipeline inventory: reduce the lead time
\end{itemize}

\subsection{ABC Inventory Planning}

A method for grouping items by dollar volume to identify those items to be monitored closely.
\begin{itemize}
	\item ``A'' items: high dollar volume (20\% item volume, 80\% dollar volume)
	\item ``B'' items: moderate dollar volume (30\% item volume, 15\% dollar volume)
	\item ``C'' items: low dollar volume (50\% item volume, 5\% dollar volume)
\end{itemize}

\begin{figure}[H]
	\centering
	\includegraphics[width=0.7\linewidth]{img/abc.png}
\end{figure}

\subsection{Characteristics of Inventory Systems}

\begin{itemize}
	\item Demand
	      \begin{itemize}
		      \item Constant v.s. Variable
		      \item Known v.s. Random
		      \item Independent v.s. Dependent
	      \end{itemize}
	\item Lead Time
	\item Review Time
	\item Excess Demand
	\item Changing Inventory
\end{itemize}

\subsection{Inventory Costs}

\begin{itemize}
	\item Holding or Carrying Costs: $h \cdot \int_{t_1} ^{t_2} l(t) \text{d}t = h\cdot \overline{l(t)}\cdot (t_2 - t_1)$,
	      where $l(t)$ is the inventory level, and $h$ is the unit cost per period,
	      comprising the following components:
	      \begin{itemize}
		      \item Storage costs (facility, insurance, tax, utilities)
		      \item Obsolescence/shrinkage costs (deprecated value)
		      \item Capital costs (opportunity costs)
	      \end{itemize}
	      \begin{figure}[H]
		      \centering
		      \includegraphics[width=0.4\linewidth]{img/inv-as-func.png}
	      \end{figure}
	      As $h$ is hard to define, we often use $H$,
	      which is the annual (or whole-period) holding cost per unit.
	      Then the total holding cost is $H\cdot \overline{l(t)}$.
	      $H$ is often expressed as ``annual interest rate'', $p\%$,
	      where $H = \text{unit cost} \times p\%$.
	\item Order Costs: $
		      \left\{
		      \begin{array}{ll}
			      S+c\cdot Q & , Q > 0 \\
			      0          & , Q = 0
		      \end{array}
		      \right.
	      $
	      \begin{itemize}
		      \item Setup cost: $S$
		      \item Purchase cost per unit: $c$
	      \end{itemize}
	\item Shortage or Stockout Costs, as a penalty.
	      It is higher than the revenue of ``what if the sale is completed'',
	      but is hard to estimate precisely.
\end{itemize}

\subsection{Basic Inventory Models}

\subsubsection{Economic Order Quantity Model (EOQ)}

Assumptions:
\begin{itemize}
	\item Demand for the product is known, constant, and uniform
	\item Lead time is constant
	\item Purchasing cost per unit is constant (no quantity discount)
	\item Setup cost is constant
	\item No back orders or stockouts
\end{itemize}

Use cases:
\begin{itemize}
	\item DO NOT use EOQ if you use the Make-To-Order strategy (fully customized)
	\item If the order size is constrained by some limitation,
	      or there are quantity discounts,
	      or if the replenishment is not instantaneous,
	      you should modify the EOQ.
\end{itemize}

The model:
\begin{figure}[H]
	\centering
	\includegraphics[width=0.7\linewidth]{img/EOQ-model.png}
\end{figure}

Total annual cost = Annual purchase cost + Annual ordering cost (purchase setup cost)
+ Annual holding cost
\[ TC = DC + \dfrac{D}{Q}S + \dfrac{Q}{2}H \]

\begin{itemize}
	\item $TC$ = Total annual cost
	\item $D$ = Annual demand in units
	\item $C$ = Cost per unit
	\item $Q$ = Quantity to be order (the variable to $\arg \min$)
	\item $S$ = Ordering cost (purchase setup cost)
	\item $H$ = Annual holding cost per unit
\end{itemize}

It can be easily identified in the EOQ Model figure that the average inventory level is
$\dfrac{Q}{2}$, therefore the annual holding cost is $\dfrac{Q}{2}H$.

Key results:
\begin{itemize}
	\item The optimization result is $EOQ = \sqrt{\dfrac{2DS}{H}}$
	\item The corresponding optimal annual variable cost
					\footnote{The part in TC that can be optimized, which is $\dfrac{D}{Q}S + \dfrac{Q}{2}H$. The $DC$ is determined and cannot be changed.} 
				is $C^* = \sqrt{2SDH}$
	\item For any $Q$, $\dfrac{C(Q)}{C^*} = \dfrac{1}{2} (\dfrac{Q}{Q^*} + \dfrac{Q^*}{Q})$
	\item Reorder point, the point in time by which stock must be ordered, is $R = \bar{d} L$,
	      where
	      \begin{itemize}
		      \item $\bar{d}$ is average demand per time period
		      \item $L$ is the number of time periods between placing order and delivery
	      \end{itemize}
	      Reorder point in inventory level is equivalent to lead time in time.
	      In the following example, reorder point is $R=1040$, and lead time is 4 months.
	      \begin{figure}[H]
		      \centering
		      \includegraphics[width=0.7\linewidth]{img/reorder-point.png}
	      \end{figure}
\end{itemize}

\subsubsection{EOQ Model with Finite Production Rate}

Considers a manufacturer that will produce an order quantity over a period of time
rather than all at once instantly.

\begin{figure}[H]
	\centering
	\includegraphics[width=0.9\linewidth]{img/EOQ-model-with-RR.png}
\end{figure}

The model:
\[
	\begin{aligned}
		TC & = DC + \dfrac{D}{Q} S + \dfrac{I_{\max}}{2}H \\
		I_{\max} &= (p-d) \dfrac{Q}{p}
	\end{aligned}
\]

\begin{itemize}
	\item $d$ = Material usage rate
	\item $p$ = Supplier production or delivery rate
	\item $(p-d)$ = Inventory accumulation rate after an order is placed
	\item $\dfrac{Q}{p}$ = number of time periods required to fill the order
\end{itemize}

Optimum EOQ = $\sqrt{
		\dfrac{2DS}{H} \cdot \dfrac{p}{(p-d)}
	}$

\subsubsection{Quantity-Discount Model}

Consider the case when there are price discounts associated with order quantities.

With the unit cost determined by the order quantity, there are two types fo discounts:
\begin{itemize}
	\item Incremental discount: The discounted unit price is imposed on the incremental unites
	\item All-units discount: The discounted unit price is imposed on all units
\end{itemize}

Visualization ($C(Q)$ is the total cost):

\begin{figure}[H]
	\centering
	\begin{minipage}[t]{0.49\textwidth}
		\centering
		\includegraphics[width=\textwidth]{img/discount-cost-incremental.png}
	\end{minipage}
	\begin{minipage}[t]{0.49\textwidth}
		\centering
		\includegraphics[width=\textwidth]{img/discount-cost-all-unit.png}
	\end{minipage}
\end{figure}

Therefore, the TC is a piecewise function of $Q$ (order size).
To determine the minimum TC, calculate minimum TC for each piece of the function, and compare their value.

\subsection{Inventory Control Systems}

\subsubsection{Continuous Review System}

The system tracks the remaining inventory of a SKU.
Each time a withdrawal is made, the system determines whether it is time to reorder.
It is a system where the order quantity remains constant but the time between orders varies
(because in the real world, the demand is varying).

Example: Always purchasing a dozen eggs when there are only two eggs left in the refrigerator.

The choice of average quantity can be determined by the EOQ model, by using average demand expectation.
However, if the choice of reorder point is also determined by the average demand,
there are 50\% of cases that you will go stock out, which is generally unacceptable.

Solution: Add safety stock. Change reorder point to average demand during lead time + safety stock.
Steps:
\begin{enumerate}
	\item Choose an appropriate service level
	\item Determine the statistical distribution of demand during lead time
	\item Determine the safety stock and reorder point, $z \sigma_{dLT}$
\end{enumerate}

The idea is that, the demand during a certain time window can be measured by a probability distribution.
If the actual demand in this window is less than the safety stock, there is no stock out.
This is described by the cdf of the above distribution.

\begin{figure}[H]
	\centering
	\includegraphics[width=0.6\linewidth]{img/safety-stock-service-level.png}
\end{figure}

Total cost of continuous review system can be measured in two ways:
\begin{itemize}
	\item annual inventory holding cost + annual ordering cost + annual safety stock holding cost
	\item annual inventory holding cost + annual ordering cost + possible backorder cost
\end{itemize}

\subsubsection{Periodic Review System / Fixed-Time Period}

Inventory position is reviewed periodically.
It is a system where the time period between orders remains constant but the order quantity varies.

Example: Always refilling the gas tank of a delivery truck at the end of each day.

Each order placed is actually the demand in the last period.
That is, after each order is placed,
the total inventory on hand and in transit is constant (T in the figure below).

\begin{figure}[H]
	\centering
	\includegraphics[width=0.8\linewidth]{img/periodic-review-system.png}
\end{figure}

\begin{itemize}
	\item Selecting the time between reviews: based on the EOQ
	\item Determine the target inventory level
	      \begin{itemize}
		      \item The protection interval is P+L, inventory-check interval plus the order lead time
		      \item T = expected demand during P+L + safety stock
		      \item This system requires more safety stock
	      \end{itemize}
	\item Total cost = annual inventory holding cost + annual ordering cost + annual safety stock holding cost
\end{itemize}

\subsubsection{Comparison of the Q and P Systems}

The advantage of P systems (periodic review system)
\begin{itemize}
	\item The system is convenient
	\item Orders for multiple items from the same supplier can be combined into a single purchase order
	\item Inventory level needs to be known only when a review is made
\end{itemize}

The advantage of Q systems (continuous review system)
\begin{itemize}
	\item The review frequency if each SKU may be individualized
	\item Fixed lot sizes, if large enough, can result in quantity discounts
	\item Lower safety stocks result in savings
\end{itemize}

\subsubsection{Hybrid Systems}

\begin{itemize}
	\item Min-max system / (s, S) system:
	      Inventory level is reviewed fixed time intervals.
	      However, an order is not necessarily placed after every review.
	      An order will be placed only after the inventory is less than a predetermined level, s.
	      The order will supplement the inventory to level S.
	      It is suitable when both review and ordering costs are high.
	\item Base stock system. A one-for-one policy, placing an order right after an inventory is sold out.
	      It is suitable for extremely expensive items.
\end{itemize}

\subsection{Perishable Inventory}

Product is only viable for sale during a single time period.
After that, the scrap value of the product is less than the initial cost of the product.

\subsubsection{Payoff Table Solution}

List the demand level and the corresponding probability, and the order quantity.
For each demand and quantity, a payoff can be calculated,
and they can be summarized based on the probability, into an expected payoff.

Choose the order quantity level with the highest expected payoff.

\subsubsection{Marginal Analysis Solution}

Let B = marginal benefit of stocking an additional unit, i.e. retail price - purchase cost.
Let C = marginal cost of stocking an additional unit, i.e. purchase price - salvage price.
The expected marginal benefit = B $\times$ P(Demand > Q),
and the expected marginal cost = C $\times$ P(Demand $\leq$ Q).

Therefore, at the optimal Q level, P(Demand $\leq$ Q) = $\dfrac{B}{B+C}$.
If Q is not continuous, the optimal Q* is $\lceil Q \rceil$.

For example, when $\dfrac{B}{B+C} = 0.925$, P(Demand $\leq$ 20) = 0.92,
and P(Demand $\leq$ 21) = 0.95.
Q is somewhere between 20 and 21, therefore Q* is 21.
The optimal purchase quantity is 21.

\subsubsection{Continuous Demand}

Begin exactly the same as the discrete demand situation,
the optimal order is the smallest value Q such that P(Demand $\leq$ Q) $\geq \dfrac{B}{B+C}$.

\subsection{Inventory Management in Services}

Services cannot stock inventory, instead, they use yield management or revenue management.
The goal is maximizing capacity utilization, selling all service capacity for a price higher than its cost.

Generally, a large proportion of capacity is sold in advance for reduced prices.
Some capacity is held for last-minute customers willing to pay full prices.

Characteristics: the ability to segment the markets and customers, high fixed costs and low variable costs.

\section{Beer Game}

Beer game is said to be out of the scope of the exam. If it is needed, refer to the slides directly.

\section{Planning for Production}

\subsection{Stages in Planning}

\begin{figure}[]
	\centering
	\includegraphics[width=0.7\linewidth]{img/stages-in-planning.png}
\end{figure}

\subsection{Concept of Independent and Dependent Demand}

\begin{itemize}
	\item Independent demand: demand for a final product, which is influenced only by market
	\item Dependent demand: demand for components, which is influenced only by final products
\end{itemize}

Master Production Schedule (MPS) is for independent demand.
Material Requirements Planning (MRP) is for dependent demand.

\subsubsection{MPS}

It details how many products will be produced within a specified periods of time.

Data related to MPS:
\begin{itemize}
	\item Constant/Initial values:
	      \begin{itemize}
		      \item Quantity on Hand
		      \item Lot Size
		      \item Lead Time
	      \end{itemize}
	\item Variables in each week:
	      \begin{itemize}
		      \item Forecast
		      \item Customer orders booked
		      \item Projected on-hand inventory
		      \item MPS quantity
		      \item MPS start
	      \end{itemize}
\end{itemize}

Steps to develop an MPS:

Calculate projected on-hand inventory, which is an estimate of the amount of inventory available each week,
after demand has been satisfied.

\[
	I_t = I_{t-1} + \text{MPS}_t - \max (F_t, CO_t)
\]

where
\begin{itemize}
	\item $I_t$ = projected on-hand inventory balance at the end of week $t$
	\item $\text{MPS}_t$ = MPS quantity \red{due} in week t
	\item $F_t$ = forecast of orders in week $t$
	\item $CO_t$ = customer orders booked for shipment in week $t$
\end{itemize}

When this formula indicates a negative $I_t$, which is a shortage signal,
a MPS should be scheduled \red{due} in week $t$.

The final MPS table:
\begin{figure}[H]
	\centering
	\includegraphics[width=0.7\linewidth]{img/filled-MPS-table.png}
\end{figure}

Note that the projected on-hand inventory is calculated from the \textbf{actual} customer orders booked
for week 1 $\rightarrow$ 2,
but is calculated from the \textbf{forecast} for other weeks.

The goal is to maintain a nonnegative projected on-hand inventory balance.
That is, as shortages are detected, MPS quantities should be scheduled to cover them.
Note that it should be planned ahead because of the lead time.

\subsubsection{MRP}

MPR specifies the replenishment schedules of all the components and raw materials of a product.

Overview:
\begin{figure}[H]
	\centering
	\includegraphics[width=0.7\linewidth]{img/MRP-overview.png}
\end{figure}

\paragraph{MRP Input: Bill of Materials (BOM)}

A record of all components of a product,
the parent-component relationship and the usage quantities and times.

\begin{figure}[H]
	\centering
	\includegraphics[width=0.7\linewidth]{img/POM-example.png}
\end{figure}

Time relationships can also be annotated on the edges in BOM to show the lead time between each level.

\paragraph{Inventory Record}

A record that shows an item's lot-size policy, lead time, and various time-phased data.

It's basically the same as MPS, except for the related variables are:
\begin{itemize}
	\item Gross requirements
	\item Scheduled receipts (initial quantity on hand)
	\item Projected on-hand inventory
	\item Planned receipts
	\item Planned order releases
\end{itemize}

The planned order releases indicates planned receipts after the lead time,
which should keep the projected on-hand inventory to be non-negative.

\paragraph{MRP Explosion}

Explode the MPS table, according to the BOM, with the Inventory Record,
to determine when and how much a raw/intermediate material should be produced.


\subsubsection{Planning Factor -- Three Kinds of Lot Size Rules}

\begin{itemize}
	\item Fixed order quantity (FOQ) rule, which maintains the same order quantity each time an order is issued,
					but the time intervals can be adjusted to ensure the projected on-hand inventory.
	\item Periodic order quantity (POQ) rule, which issues the order for predetermined time intervals,
					but the order quantity (lot size) can be adjusted to ensure the projected on-hand inventory.
  \item Lot-for-Lot (L4L) rule, where the lot size covers the gross requirements of a single week.
					This is a special case of one-week periodic order quantity.
\end{itemize}

The average inventory level of FOQ is the largest, and that of POQ is smaller,
and L4L is a just-in-time production so that the average inventory level is 0.

\subsubsection{Planning Factor -- Lead Time}

Lead time is an estimate of the time between placing an order and receiving the item in inventory.
It consists time estimations for:

\begin{itemize}
	\item Setup time
	\item Processing time
	\item Materials handling time between operations
	\item Waiting time
\end{itemize}

\subsubsection{Planning Factor -- Safety Stock}

Safety stock is to protect against uncertainties, including lead time,
item quality, etc.

Using rule is that the planned receipt should be scheduled whenever the projected on-hand inventory
drops below the desired safety stock level.

\subsection{Management Issue}

\subsubsection{MPS Frozen and ATP}

The MPS table may variate because the forecast and customer orders booked are not stable,
it will cause the projected MPS start to change frequently.

After confirming the MPS, it should be frozen.
The Available-to-Promise (ATP) inventory,
can then be calculated on this basis.
ATP is the quantity produced by an MPS, or the quantity on hand,
that marketing department can promise to deliver on specific dates.

Calculation of ATP:
\begin{itemize}
	\item The first week: current on-hand inventory + MPS quantity this week -
					the cumulative total of booked orders before the week in which the next MPS quantity arrives
  \item Subsequent weeks: Whether an MPS is scheduled to be completed?
	\begin{itemize}
		\item No: no need to calculate again by definition.
		\item Yes: that week's MPS quantity - the cumulative total of booked orders from that week
						to the the week in which the next MPS quantity arrives.
	\end{itemize}
\end{itemize}

Example:
\begin{figure}[H]
	\centering
	\includegraphics[width=0.75\linewidth]{img/atp-calc.png}
\end{figure}

The first ATP, 17, is quantity on hand - customer orders booked.
The second ATP, 91, is $150-27-24-8$.
The third one is just that MPS quantity, because no customer orders have been booked at that time.

The total amount that marketing department can promise to deliver on specific dates
is the summation of the ATP before that specific date.
For example, $91+17=108$ items can be promised to deliver in Week 2, or Week3.
$17+91+150=258$ items can be promised to deliver in Week 7, or Week 8.

\subsubsection{Importance of Data}

The input of MPS, BOM, inventory records, and lead time, must be precise,
otherwise the plan will not work.

\subsubsection{Evolution from MRP to MRPII}

The CRP (Capacity Requirement Planning) system is added to make sure that
the plan generated by the MRP system is balanced and available.

MRPII (Manufacturing Resources Planning) is a system that ties the basic MRP system to the company's financial system,
and other processes.
It will generate a performance report based on the output of MRP and the cost data.

\subsubsection{Evolution from MRPII to ERP}

The ERP system combines the MRPII and customer service, accounting and finance, HR, data analysis, sales and marketing, supply chain management system.
It is the most confound information system in today's company.

\subsubsection{Resource Planning for Service Providers}

Focus on the capacity, and use bill of resources instead of bill of materials.
Other considerations may vary as well.

\section{Planning for Project}

A project is an interrelated set of activities with a \red{definite starting and ending point}
which results in a \red{unique outcome} for a \red{specific allocation of resources}.

The goal of project management is to:
\begin{itemize}
	\item Complete the project on time or earlier
	\item Do not exceed the budget
	\item Meet the specifications
\end{itemize}

Four major phases of project management:
\begin{enumerate}
	\item Defining
		\begin{itemize}
			\item Define the scope and objectives
			\item Select the project manager and team
			\item Recognizing organization structure
		\end{itemize}
	\item Planning
		\begin{itemize}
			\item Define the Work Breakdown Structure (WBS)
			\item Diagramming the network
			\item Developing the schedule
			\item Analyzing cost-time trade-offs
			\item Assessing risks
		\end{itemize}
	\item Execution
	\item Close out
\end{enumerate}

\subsection{Defining and Planning}

\subsubsection{Defining Work Breakdown Structure}

Similar to BOM, break a large task into various small tasks in a table, or a tree.
Note that the precedence relationship should also be established and the estimated time should be given.

\subsubsection{Diagramming the Network}

Based on the immediate predecessor, a DAG can be depicted to show the predecessor relationship,
and the time required for each task can be marked as well.
\begin{figure}[H]
	\centering
	\includegraphics[width=0.9\linewidth]{img/DAG.png}
\end{figure}

\subsubsection{Developing the Schedule}

The completion time of a project can be found by finding the critical path.

Each node (activity) have five attributes:
\begin{itemize}
	\item Earliest start time
	\item Earliest finish time
	\item Latest start time
	\item Latest finish time
	\item Activity duration
\end{itemize}

The latest start/finish time means latest time for the task to start which will not cause the whole project to postpone.

The earliest start/finish time can be computed from the start point,
the latest start/finish time can be computed backwards from the ending point.
The difference between the earliest and latest time is the slack time.
The path without slack time is the critical path.

\subsubsection{Analyzing Cost-Time Trade-offs}

Project costs:
\begin{itemize}
	\item Direct costs: labor, materials, etc. Costs directly related to project activities.
	\item Indirect costs: administration, depreciation, financial, etc. Anything that can be avoided by reducing total project time.
	\item Penalty costs: incurred if the project extends beyond some specific date.
\end{itemize}

Project crashing (or expediting): Shortening some activities within a project to reduce overall project completion time

Crash time: The shortest possible time to complete an activity.
Crash cost: The total cost of an activity which is to be finished in its crash time (will be higher than its normal cost).

Maximum time reduction: Normal time - Crash time.
Cost of crashing per week: (Crash cost - Normal cost) / Maximum time reduction

The procedures:
\begin{enumerate}
	\item Find the activity \red{on the critical path} with the lowest cost of crashing per week
	\item Reduce its time (crash it) until
	\begin{itemize}
		\item It cannot be further reduced
		\item Another path becomes critical
		\item The increase in the cost exceeds the savings that result from shortening the project
	\end{itemize}
	\item Repeat until no possible improvement
\end{enumerate}


\subsubsection{Assessing Risks}

The uncertainty can be estimated with:
\begin{itemize}
	\item The optimistic time $a$
	\item The most likely time $m$
	\item The pessimistic time $b$
\end{itemize}

Each activity time in a project is modelled as a random variable derived from a beta probability distribution.

For beta distribution, the average (mean finish time) is $\dfrac{a+4m+b}{6}$,
and the variance is $\left(\dfrac{b-a}{6}\right)^2$.

In this setup, the time used for creating the DAG (and thus identifying the critical path) is the \textbf{mean time}
calculated from the formula above, \red{not the most likely time}.

Suppose that the activity duration times are i.i.d.,
according to the central limit theorem,
$T_E$ is the sum of mean times of activities on the critical path.
$\sigma_p^2$ is the sum of the variance of activities on the critical path.

Use the $z = \dfrac{T- T_E}{\sigma_p}$ transformation where $T$ is the due date for the project,
we can test the probability that the project will not overdue.

\subsection{Execution and Close Out}

The gantt chart can be used for monitoring and controlling schedule status and project resources.

\section{Lean System}

Definition: operation systems that maximize the value added by each of a company's activities through removing waste from them,
where ``waste'' is the activities that only increase the cost but does not add value.

\subsection{JIT Production}

JIT production is a pull system instead of a push system.
It utilized the Kanban system, in which:
\begin{itemize}
	\item number of units held by each container: size of the production lot
	\item number of authorized containers: WIP inventory + safety stock
	\item number of Kanbans: number of containers
\end{itemize}

\[
	K = \dfrac{d}{c}(w+p)(1+\alpha)
\]

\begin{itemize}
	\item K = number of Kanbans
	\item d = expected daily demand for the part, in units
	\item w = average waiting time plus materials handling time per container
	\item p = average processing time per container
	\item c = quantity in a standard container of the part
	\item $\alpha$ = safety stock coefficient
\end{itemize}

\subsection{Labour Productivity}

One Worker, Multiple Machines (OWMM)

The production assembly uses a ``U'' shape instead of the traditional straight line,
so that a worker can easily manipulate the machines on both sides.

\subsection{Quality Control}

\begin{itemize}
	\item Precaution and control is more important and costly effective than examine the flawed products
	\item All workers should do quality control by themselves, instead of forming a specialized QA department.
					The workers' self-quality-control team is called QC team.
\end{itemize}

\subsection{Continuous Improvement}

Philosophy: Any organization has room for improvement under any circumstances.

The most important things are ``continuous'' and ``employee participation''.




%-------%



\end{document}



