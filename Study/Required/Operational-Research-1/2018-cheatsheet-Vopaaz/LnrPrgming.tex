% Chinses Article template created by Vopaaz
\documentclass[UTF8, 6pt]{ctexart}
\usepackage{geometry}
\geometry{a4paper, scale=0.99, nohead, nofoot}
\usepackage{enumerate}
\usepackage{enumitem}

\setenumerate[1]{itemsep=0pt,partopsep=0pt,parsep=0pt ,topsep=0pt}
\setitemize[1]{itemsep=0pt,partopsep=0pt,parsep=0pt ,topsep=0pt}
\setenumerate[2]{itemsep=0pt,partopsep=0pt,parsep=0pt ,topsep=0pt}
\setitemize[2]{itemsep=0pt,partopsep=0pt,parsep=0pt ,topsep=0pt}
\setdescription{itemsep=0pt,partopsep=0pt,parsep=0pt ,topsep=0pt}

\usepackage{setspace}
\usepackage{hyperref}
\hypersetup{colorlinks, allcolors = black}

\usepackage{amsmath}
\usepackage{amssymb}

\usepackage{multicol}
\usepackage{multirow}

\usepackage{color}
\usepackage{graphicx}

\newcommand{\st}{\text{s.t.}}

\pagestyle{empty}

% Specifically necessary to be changed

\usepackage{titlesec}

\titleformat{\section}
{\normalfont\fontsize{10}{12}\bfseries}{\thesection}{1em}{\mcompress}

\newcommand{\mcompress}{\vspace{-10 px}}


\newcommand{\sectionline}{\rule[2pt]{0.45\textwidth}{0.05em}}

\newcommand{\reflecttable}{
\begin{table}[htb]
	\centering
	\footnotesize
	\begin{tabular}{|l|l|l|}
		\hline
		对应项目                          & 原问题         & 对偶问题        \\ \hline
		目标函数类型                        & max         & min         \\ \hline
		\multirow{2}{*}{目标函数系数与约束右边项} & 目标函数系数      & 右边项系数       \\
		                              & 右边项系数       & 目标函数系数      \\\hline
		\multirow{2}{*}{变量与约束}        & 变量数 n       & 约束数 n       \\
		                              & 约束数 m       & 变量数 m       \\\hline
		\multirow{6}{*}{变量类型与约束类型}    & 变量 $\geq 0$ & 约束 $\geq$   \\
		                              & 变量 $\leq 0$ & 约束 $\leq$   \\
		                              & 变量无限制       & 约束 $=$      \\
		                              & 约束 $\geq$   & 变量 $\leq 0$ \\
		                              & 约束 $\leq$   & 变量 $\geq 0$ \\
		                              & 约束 $=$      & 变量无限制 \\\hline
	\end{tabular}
\end{table}
}

\usepackage{multicol}

\begin{document}

\scriptsize

\twocolumn

\section{单纯形法}

线性规划问题的标准形式:

\mcompress
\[
\begin{aligned}
\max z &= CX \\
\st\ AX&=b\\
X& \geq 0
\end{aligned}
\]
\mcompress


要点:

\mcompress

\begin{multicols}{2}

\begin{itemize}
	\item 目标函数求最大
	\item 所有约束写成等号(通过松弛变量等)
	\item 所有决策变量大于等于0
	\item $a_{ij}$ 的符号无限制,\textbf{$b_j$ 必须为正}
\end{itemize}
\end{multicols}

\mcompress
\sectionline

单纯形表

例题:

\mcompress
\[
\begin{array}{lrrrrrl}
	\max z & =  3x_1 & -3x_2 &       &   +5x_4 & -x_5 &       \\
	\st    &     x_1 &       & -2x_3 &   +2x_4 &      & =12   \\
	       &         &   x_2 & -2x_3 &         &      & =1    \\
	       &         &       & -4x_3 &   +3x_4 & +x_5 & =27   \\
	       &         &       &  x_1, & \ldots, &  x_5 & \geq0
\end{array} 
\]
\mcompress

初始单纯形表

\mcompress
\[
\begin{array}{ccc|ccccc}
	    & c_j \rightarrow &    &  3   & -3  &  0  &  5  & -1  \\\hline
	C_B &       X_B       & b  & x_1  & x_2 & x_3 & x_4 & x_5 \\ \hline
	 3  &       x_1       & 12 &  1   &  0  & -2  & [2] &  0  \\
	-3  &       x_2       & 1  &  0   &  1  & -2  &  0  &  0  \\
	-1  &       x_5       & 27 &  0   &  0  & -4  &  3  &  1  \\ \hline
	    &    c_j - z_j    &    &  0   &  0  & -4  &  2  &  0  \\ \hline
	 5  &       x_4       & 6  & 1/2  &  0  & -1  &  1  &  0  \\
	-3  &       x_2       & 1  &  0   &  1  & -2  &  0  &  0  \\
	-1  &       x_5       & 9  & -3/2 &  0  & -1  &  0  &  1  \\ \hline
	    &                 & 18 &  -1  &  0  & -2  &  0  &  0
\end{array}
\]
\mcompress

上表中有关的说明:

\begin{itemize}
	\item 全表第一行为目标函数的系数
	\item ``b" 对应的一列为约束的系数,在后续进行线性变换时,\textbf{要}跟着变换。另注意标准型要求了 $b>0$.
	\item $C_B$ 一列为基变量 $X_B$ 对应的系数,\textbf{不}跟随主矩阵进行线性变换,而是在换基迭代完成后更新基变量对应的系数
	\item $c_j - z_j$ 一行的具体计算方法为 $c_j - \sum \limits_{i=1}^{m}c_i a_{ij}$, 例如:$-4 = 0 - 3 \times (-2) - (-3)\times (-2) - (-1)\times (-4)$. 一个技巧为:基变量的此数值必然为 0.
	\item 当 $c_j - z_j$ 一行全部为负数时,即为最终解。若存在正数,则将其中\textbf{最大的正数}对应的作为入基变量
	\item 确定入基变量后,可确定出基变量。指标为入基变量列对应的\textbf{正 b/a 中的最小者},即:
	\begin{enumerate}
		\item 确定了入基变量为 $x_4$
		\item 对 $x_1$: $12\div 2 = 6$
		\item 对 $x_2$: 除数为 0,跳过
		\item 对 $x_3$: $27 \div 3 = 9$
		\item 选取其中最小的正数为 6,对应 $x_1$, 因此 $x_1$ 为出基变量
		\item 最终目标为通过线性变换令 [2] 变为 1, 且其同一列的其他值为 0
	\end{enumerate}
\item 最终的目标函数值为 18,计算方法为在最终的单纯形表中,计算 $C_B^T \cdot b$, 即 $5\times 6 +(-3) \times 1 + (-1)\times 9 = 18$
\end{itemize}

\sectionline

以符号形式表示的通式:

\mcompress
\[
\begin{array}{ccc|cc}
	    &     &            &      C       &     0      \\\hline
	    & X_b &            &      X       &    X_s     \\ \hline
	 0  & X_s &     b      &      A       &     I      \\ \hline
	C_B & X_B &  B^{-1}b   &   B^{-1}A    &   B^{-1}   \\ \hline
	    &     & C_BB^{-1}b & C-C_BB^{-1}A & -C_BB^{-1}
\end{array}
\]
\mcompress

有关的说明:

\begin{itemize}
	\item 这里假设了所有原始问题中的约束都是不等号,松弛变量直接生成了单位矩阵
\end{itemize}

\sectionline

初始问题无法找到一个单位阵的解决方案

(一)人工变量法(大 M 法)

在等式中引入适当的人工变量使初始单位矩阵存在,并在目标函数中添加 $-M x_i$ ($x_i$ 是添加的人工变量) 项,将 $M$ 认为是一个无穷大的数。用这种方法进行换基迭代,如果最终添加的人工变量留在了基内,那就认为这个问题没有可行解。


(二)两阶段法

第一阶段,不处理原问题,而是处理去除人工变量的新问题:

目标函数:最小化所有人工变量之和(转化为最大化所有人工变量的相反数的和),与原目标函数无任何关联

约束:与大 M 法相同,在原约束的基础上增加人工变量,使得存在初始单位矩阵。

若第一阶段使得所有人工变量全部成为非基变量,则进入第二阶段,否则无可行解。

第二阶段,在上一阶段结果的基础上去除所有的人工变量,并且恢复原始的目标函数,继续计算。

例题:转化为标准形式的原始问题为:

\mcompress
\[
\begin{array}{lrrrrrl}
	\max z & =-3x_1 &      & +x_3 &         &      &        \\
	\st    &    x_1 & +x_2 & +x_3 &    +x_4 &      & =4     \\
	       &  -2x_1 & +x_2 & -x_3 &         & -x_5 & =1     \\
	       &        & 3x_2 & +x_3 &         &      & =9     \\
	       &        &      & x_1, & \ldots, &  x_5 & \geq 0
\end{array}
\]
\mcompress

大 M 法(在此基础上完全按照单纯形法逻辑进行计算即可):

\mcompress
\[
\begin{array}{ccc|ccccccc}
	    & c_j \rightarrow &   & -3  &  0  &  1  &  0  &  0  & -M  & -M  \\\hline
	C_B &       X_B       & b & x_1 & x_2 & x_3 & x_4 & x_5 & x_6 & x_7 \\ \hline
	 0  &       x_4       & 4 &  1  &  1  &  1  &  1  &  0  &  0  &  0  \\
	-M  &       x_6       & 1 & -2  &  1  & -1  &  0  & -1  &  1  &  0  \\
	-M  &       x_7       & 9 &  0  &  3  &  1  &  0  &  0  &  0  &  1 
\end{array}
\]
\mcompress

两阶段法,第一阶段:

\mcompress
\[
\begin{array}{ccc|ccccccc}
	    & c_j \rightarrow &   &  0  &  0  &  0  &  0  &  0  & -1  & -1  \\\hline
	C_B &       X_B       & b & x_1 & x_2 & x_3 & x_4 & x_5 & x_6 & x_7 \\ \hline
	 0  &       x_4       & 4 &  1  &  1  &  1  &  1  &  0  &  0  &  0  \\
	-1  &       x_6       & 1 & -2  &  1  & -1  &  0  & -1  &  1  &  0  \\
	-1  &       x_7       & 9 &  0  &  3  &  1  &  0  &  0  &  0  &  1
\end{array}
\]
\mcompress

据此计算最终将会得到

\mcompress
\[
\begin{array}{ccc|ccccccc}
	C_B &   X_B   & b & x_1 & x_2 & x_3 & x_4 & x_5  & x_6  & x_7  \\ \hline
	 0  &   x_4   & 0 &  0  &  0  &  0  &  1  & -1/2 & 1/2  & -1/2 \\
	 0  &   x_2   & 3 &  0  &  1  & 1/3 &  0  &  0   &  0   & 1/3  \\
	 0  &   x_1   & 1 &  1  &  0  & 2/3 &  0  & 1/2  & -1/2 & 1/6  \\ \hline
	    & c_j-z_j &   &  0  &  0  &  0  &  0  &  0   &  -1  &  -1
\end{array}
\]
\mcompress

第二阶段的开始:

\mcompress
\[
\begin{array}{ccc|ccccc}
	    & c_j \rightarrow &   & -3  &  0  &  1  &  0  &  0   \\ \hline
	C_B &       X_B       & b & x_1 & x_2 & x_3 & x_4 & x_5  \\ \hline
	 0  &       x_4       & 0 &  0  &  0  &  0  &  1  & -1/2 \\
	 0  &       x_2       & 3 &  0  &  1  & 1/3 &  0  &  0   \\
	-3  &       x_1       & 1 &  1  &  0  & 2/3 &  0  & 1/2  \\
\end{array}
\]
\mcompress
	
此后完全按照正常逻辑即可。

\sectionline

无可行解:在所有检验数 $\leq 0$ 的情况下,基变量中含有非零的人工变量

无穷多最优解:在所有检验数 $\leq 0$ 的情况下,存在非基变量的检验数为 0

\noindent
无界解:对某一检验数 $>0$, 其对应的 $P_j$ (主体矩阵对应列) 中的所有系数均 $\leq 0$

\sectionline

\mcompress
\mcompress

\section{对偶理论}

原问题与对偶问题的标准形式:
\mcompress
\mcompress
\begin{multicols}{2}
\[
\begin{aligned}
\max z &= CX \\
\st\ AX&\leq b\\
X& \geq 0
\end{aligned}
\]

\[
\begin{aligned}
\min w &= Y^Tb \\
\st\ A^TY&\geq C^T\\
Y& \geq 0
\end{aligned}
\]
\end{multicols}
\mcompress

对非标准情况,具体的变化见下表。

\mcompress
\reflecttable
\mcompress

例如,一个具体的原问题和对偶问题分别为:


\centerline{
\begin{tabular}{l|l}
\noindent
$
\begin{array}{rrrl}
	\max z & = 2x_1 &    +x_2 &  \\
	       &   5x_2 & \leq 15 &  \\
	6x_1   & + 2x_2 & \leq 24 &  \\
	x_1    &   +x_2 &  \leq 5 &  \\
	x_1,   &    x_2 &  \geq 0 &
\end{array}
$
&
\noindent
$
\begin{array}{rrrrl}
	\min w & = 15y_1 & +24y_2 & + 5y_3 &  \\
	       &    6y_2 &   +y_3 & \geq 2 &  \\
	5y_1   &  + 2y_2 &   + y3 & \geq 1 &  \\
	y_1,   &     y_2 ,&    y_3 & \geq 0 &
\end{array}
$
\end{tabular}
}

对其按照标准方法进行单纯形法计算可分别得到:

\mcompress
\begin{table}[htb]
	\centering
	\footnotesize
	\begin{tabular}{|cc|cc|ccc|}
		\hline
		      &                         &  \multicolumn{2}{c}{原变量}   & \multicolumn{3}{|c|}{原松弛变量} \\ \hline
		      &                         & $x_1$ &       $x_2$        & $x_3$ & $x_4$ &    $x_5$    \\ \hline
		$x_3$ &          15/2           &   0   &         0          &   1   &  5/4  &    -15/2    \\
		$x_1$ &           7/2           &   1   &         0          &   0   &  1/4  &    -1/2     \\
		$x_2$ &           3/2           &   0   &         1          &   0   & -1/4  &     3/2     \\ \hline
		\multicolumn{2}{|c|}{$c_j-z_j$} &   0   &         0          &   0   &  1/4  &     1/2     \\ \hline
		      &                         & \multicolumn{2}{c}{对偶剩余变量} & \multicolumn{3}{|c|}{对偶变量}  \\ \hline
		      &                         & $y_4$ &       $y_5$        & $y_1$ & $y_2$ &    $y_3$ \\ \hline
	\end{tabular}
\end{table}
\mcompress\mcompress
\begin{table}[htb]
	\footnotesize
	\centering
	\begin{tabular}{|cc|ccc|cc|}
		\hline
		      &                       & \multicolumn{3}{c}{对偶变量}  & \multicolumn{2}{|c|}{对偶剩余变量} \\\hline
		      &                       & $y_1$ & $y_2$ &   $y_3$   & $y_4$ &       $y_5$        \\\hline
		$y_2$ &          1/4          & -5/4  &   1   &     0     & -1/4  &        1/4         \\
		$y_3$ &          1/2          & 15/2  &   0   &     1     &  1/2  &        -3/2        \\\hline
		\multicolumn{2}{|c|}{$c_j-z_j$} & 15/2  &   0   &     0     &  7/2  &        3/2         \\\hline
		      &                       & \multicolumn{3}{c}{原松弛变量} &  \multicolumn{2}{|c|}{原变量}   \\ \hline
		      &                       & $x_3$ & $x_4$ &   $x_5$   & $x_1$ &       $x_2$\\ \hline
	\end{tabular}
\end{table}

\mcompress

注意原变量/松弛变量,约束右边项/最终检验系数,以及A矩阵之间负转置的关系。

对偶性质:
\begin{itemize}
	\item 原问题最优解的目标函数值是其对偶问题目标函数值的下界;反之是上界
	\item 原问题有最优解,则对偶问题有最优解且最终目标函数相等,因此在上例子中,可以只通过一个表读出两个问题的目标函数最优解
	\item 原问题无界,则对偶问题无可行解
	\item 原问题无可行解,则对偶问题无界
	\item \textbf{互补松弛性质}:在原问题最优解中,如果某一约束条件对应的对偶变量值 $\neq 0$, 则这约束取严格等式;反之如果约束条件取严格不等式,则对应的对偶变量必为0.
\end{itemize}

一种约束的影子价格:其在对偶问题中对应的变量在最优解时候的值。

\textbf{单位为:目标函数单位/资源单位}

\textbf{意义为}:

\begin{itemize}
	\item 保持其他条件不变
	\item 资源在【算出来的一定区间之内】变化
	\item 增加一单位资源(约束右边项 b)能够增加的目标函数值
\end{itemize}

\sectionline

对偶单纯形法:

标准形式:其余要求与原标准形式相同,但是不要求 $b$ 的正负性,但要求初始单纯形表中的所有检验系数 $c_j-z_j$ 必须全部为负数。

\begin{itemize}
	\item 如果所有 $b\geq 0$,则表明问题已经达到最优解,否则取负的 $b$ 中\textbf{最小的}(绝对值最大的)作为出基变量
	\item 在出基变量对应行中,如果所有的 $a_{ij} \geq 0$, 则原问题无可行解。否则,对于所有\textbf{为负数}的 $a_{ij}$, 选其中 $\dfrac{c_j-z_j}{a_{ij}}$ 的\textbf{最小者}对应的变量为入基变量
\end{itemize}

\sectionline

\noindent
灵敏度分析(以下 $B^{-1}$ 是最终单纯形表中,位于初始单纯形表中单位矩阵位置的数字):

$c_j$ 变化:更换最终单纯形表中 $c_j$, 并正常继续计算,\textbf{注意开头行和基列都要更换}

$b_j$ 变化:运用公式 $\Delta b' = B^{-1} \Delta b$,将 b 的变化反映到最终的单纯形表中

增加一个变量 $x_j$:计算 $P'_j = B^{-1}P_j$, 并入原表最右侧后计算 $\sigma'_j = c_j-z_j$

参数 $a_{ij}$ 变化:将发生变化的那一列看作是一种新的产品,按上一条的方法并在原列的右侧并删去原列。若原列为基变量, $P'_j$ 可能不再符合基变量的要求。此时通过线性变换将其变换至符合要求($b$ 以及 $A$ 矩阵的其他部分也跟着变化)。此后再计算时,有可能 $b \geq 0 $ 与 $c_j-z_j \leq 0$ 均不满足。此时,将 $b$ 为负的行两端乘以 $-1$, 并加上一个人工变量(原单位矩阵的 1 变为了 -1, 因此需要人工变量再生成一个 1),在目标函数中其权重为 $-M$, 随后继续通过单纯形法继续计算。 

增加一个约束条件:首先,将原问题最优解代入新的约束,如果满足则最优解不变。否则,将新的约束直接作为新的一行添加入。之后,原为单位矩阵的列很可能不再符合要求(新的一行不为 0),要进行线性变换(将新加入一行变为 0)使其符合要求。

参数线性规划:令 $\lambda = 0$ 算出最终单纯形表,之后再将 $\lambda$ 反映到最终的单纯形表中,再进行上述灵敏度分析类似的计算。

\sectionline

当多个目标函数系数 $c_j$ 同时发生变化时,\textbf{使用 100\% 法则。}

每个系数 $c_j$ 单独变化时可以算出其最优解保持不变时允许的变化范围 $L_j \leq c_j \leq U_j$.
定义比率系数$r_{j}=\left\{\begin{array}{ll}{\Delta c_{j} /\left(U_{j}-c_{j}\right)} & {\Delta c_{j} \geq 0} \\ {\Delta c_{j} /\left(L_{j}-c_{j}\right)} & {\Delta c_{j} \leq 0}\end{array}\right.$, 
其中 $\Delta c_j$ 为问题中发生的实际增减量。

对于所有发生变化的 $c_j$, 如果 $\sum r_j \leq 1$, 则最优解必然保持不变。反之有可能变化(但是也非必然变化)。


当多个约束右边项系数 $b_i$ 同时变化时,\textbf{使用 100\% 法则。}

每个系数 $b_i$ 单独变化时可以算出其最优解保持不变时允许的变化范围 $L_{i} \leq b_{i} \leq U_{i}$.

定义比例系数 $r_{i}=\left\{\begin{array}{ll}{\Delta b_{i} /\left(U_{i}-b_{i}\right)} & {\Delta b_{i} \geq 0} \\ {\Delta b_{i} /\left(L_{i}-b_{i}\right)} & {\Delta b_{i} \leq 0}\end{array}\right.$, 其中 $\Delta c_j$ 为题中实际发生的增减量。

对于所有发生变化的 $r_i$, 如果 $\sum r_i \leq 1$, 则所有的影子价格均保持不变。反之有可能变化(但是也非必然变化)。




\sectionline

\mcompress
\mcompress

\section{运输问题}

基本问题标准形式的要求:产销平衡。基变量的个数应当始终保持 $m+n-1$ 个。产地竖向排列,销地横向排列。表中是费用,优化目标是总费用最小化。

给出运输问题的初始基可行解(Vogel 法过于复杂,略去):

\begin{itemize}
	\item 最小元素法,首先尽可能地填充运输费用最小的 cell
	\item 西北角法,首先尽可能地填充左上角的 cell
\end{itemize}

所有被填充有数字的就作为基可行解。注意基变量的个数始终保持 $m+n-1$ 个,如果不足则在适当位置填充 0.

定义:一个空格(非基变量)的闭回路,是除了它以外均为填有数字的格(基变量),由水平和竖直线段连接起来的封闭回路。(标记为 1-2-3...-1)

解的最优性检验:

\begin{itemize}
	\item 闭回路法:对于一个闭回路, 奇数位置的 $c_{ij}$ (运输费用)之和(包括 1 点出发点本身,但是只计算一次)减去偶数位置的 $c_{ij}$ 之和,如果均 $\geq 0$, 则已经达到最优解。
	\item \textbf{对偶变量法(位势法)}:一个产地(行)对应一个 $u_i$, 一个销地(列)对应一个 $v_j$, 假设任意 $u_i$ 为 0, 对每一个基变量所在的 $c_{ij}$, 有 $u_i + v_j = c_{ij}$.
	此后,对所有的非基变量,通过 $c_{ij} - u_i - v_j$ 算出检验数。
	如果均 $\geq 0$, 则已为最优解。
\end{itemize}

解的改进:

\begin{enumerate}
	\item 选定检验系数为负数且最小(绝对值最大)的 $x_{ij}$ 为入基变量
	\item 对于入基变量的闭回路,找出偶数顶点上 $x_{ij}$ 最小的那一个
	\item 将整条回路上所有奇数顶点增加这个数字,所有偶数顶点减去这个数字
\end{enumerate}

注意如果有两个点同时减为了 0, 只能将一个作为空点(打$\times$),其他的必须写 0 仍然作为基变量,保证基变量个数始终为 $m+n-1$.

产销不平衡的情况下,假想存在一个产地/销地,其数量可以使得问题变为产销平衡。同时设其单位运价为 \textbf{0}. 

销地存在最高最低需求的情况:
\begin{enumerate}
\item 假设存在一个产地 Dummy, 其产能可以使得总产能满足所有销地的最高需求
\item 将每个销地拆分为 min \& \textbf{extra} 两个销地
\item Dummy 对 min 的运输费用为 M, 对 extra 的运输费用为 0.
\end{enumerate}

\sectionline

\mcompress
\mcompress

\section{目标规划}

对每个决策目标,引入正、负偏差变量 $-d^+$ 和 $+d^-$, 分别表示决策值超过或者不足目标值的部分。$d^+ \geq 0, d^- \geq 0, d^+ \cdot d^- = 0$.

所有约束都写成目标约束的形式。

目标:使某些偏差变量\textbf{之和}最小。要求恰好达到目标值,$\min P_i(d^+ + d^-) $;要求不超过目标值 $\min P_i(d^+) $;要求不低于目标值 $\min P_i(d^-) $.通过优先因子 $P_l$ 来反映约束满足需求的优先级 $P_l \gg P_{l+1}$. 

检验数分列的单纯形法:
由于目标函数系数只有 $P_l$, 因此检验数必然是 $P_l$ 的线性组合。因此可将其写成类似下表的形式。注意这只是对目标函数种各偏差变量之和的间接写法。

\mcompress
\[
\begin{array}{ccc|cccc}
	       & c_j \rightarrow &        &      0       &  P_1   &  P_2   &  P_3   \\ \hline
	 C_B   &       X_B       &   b    &     x_i      & d_1^+  & d_2^-  & d_3^+  \\ \hline
	\vdots &     \vdots      & \vdots &    \vdots    & \vdots & \vdots & \vdots \\ \hline
	       &    c_j - z_j    &  P_1   & \hdotsfor{4}  \\
	       &                 &  P_2   & \hdotsfor{4}  \\
	       &                 &  P_3   & \hdotsfor{4} 
\end{array}
\]
\mcompress

首先注意这种形式下,目标函数为求 $\min$, 因此要求检验数\textbf{大于} 0 时才达到最优,若有小于 0 的则,则选择其中最小的(绝对值最大)要换基迭代。
之后,只要首先看 $P_1$ 一行,如果全部满足大于 0,再确认 $P_2$ 一行,以此类推。如果 $P_1$ 一行已经无法满足,那么就不需要再确认 $P_2$ 行。

变种:要求约束为获利最大。写成 $\text{revenue} - d^+ + d^-  = M$, 目标函数项为 $\min d^-$.


\sectionline
\mcompress\mcompress
\section{整数规划}

割平面法:
对松弛问题(不限制整数,其他不变)最优表中分数部分最大的 $b_i$,选取此行构造约束

$\sum_j \left[-(a_{ij} - \lfloor a_{ij} \rfloor)x_j \right] + x_{n+1} = - (b_i - \lfloor b_i \rfloor) $,其中 $x_{n+1}$ 为新松弛变量,将此约束加入单纯形表中继续计算并迭代,直到 $b$ 全部为整数为止。
($\lfloor \rfloor$ 代表向下取整。无论 $t$ 的正负, $- (t - \lfloor t \rfloor)$ 必然要是负数。)

\mcompress
\[
\begin{array}{ccc|ccccc}
	    & c_j \rightarrow &      &  3  & -1  &  0   &  0  &  0   \\ \hline
	C_B &       X_B       &  b   & x_1 & x_2 & x_3  & x_4 & x_5  \\ \hline
	 3  &       x_1       & 13/7 &  1  &  0  & 1/7  &  0  & 2/7  \\
	-1  &       x_2       & 9/7  &  0  &  1  & -2/7 &  0  & 3/7  \\
	 0  &       x_4       & 31/7 &  0  &  0  & -3/7 &  1  & 22/7 \\ \hline
	    &     c_j-z_j     &      &  0  &  0  & -5/7 &  0  & -3/7
\end{array}
\]
\mcompress


各 $b$ 中最小为第一行 $13/7$, 分数部分 $6/7$, 新的割平面约束为 $-1/7 x_3 - 2/7 x_5 + x_6 = -6/7$,将其加入单纯形表新的一行后继续计算。

分支定界法:

\begin{enumerate}
	\item 解松弛问题,如果无可行解或已经是整数解则结束
	\item 随便挑一个整数解作为剪枝的下界
	\item 对于松弛问题解中非整数的 $x_j = b_j$,分支两个后续问题,分别添加约束 $x_j \leq \lfloor b_j \rfloor$, $x_j \geq \lfloor b_j \rfloor +1$
	\item 一旦分支问题不可行或者最优解的目标值小于剪枝下界,则剪枝;否则继续迭代。
\end{enumerate}

隐枚举法(针对 01 变量):

列一张三大列的表,列分别为【$x_i$ 为01枚举| $z$ 值|各约束条件是否满足】

首先算出所有的 $z$ 值,然后从第一个开始,检查约束条件是否满足,跳过 $z$ 值比当前 valid 的最大 $z$ 值小的 $x_i$ 组合

指派问题:

\noindent
标准形式:n个人,n件事,一一对应,1代表指派,0反之;系数矩阵代表\textbf{费用,目标是最小化费用和}

匈牙利解法(以下$\textcircled{0}$代表独立零元素, $\emptyset$ 代表其他零元素):

\begin{enumerate}
	\item 对每一行减去此行最小元素,对每一列减去此列最小元素
	\item 变换后的矩阵中确定独立零元素,若有 $n$ 个则为最优解,独立零元素的位置为 1,其他为 0
	\item 若独立零元素小于 $n$ 个:对没有 \textcircled{0} 的行打勾;在打勾行中对 $\emptyset$ 所在列打勾;在打勾的列中对 \textcircled{0} 打勾;重复以上两步直到没有任何可以打勾的行或列;对\textbf{没有}勾的行划横线,对有勾的列划竖线
	\item 在没有被直线覆盖的元素中选出一个最小的。将没有被直线覆盖的行减去这个最小元素。此时直线覆盖的元素中将出现负元素,对其所在列加上这一最负元素相反数(使其为零),返回步骤2.
\end{enumerate}

确定独立零元素:在只有一个零元素的行(或列)中将此 0 变为 \textcircled{0}, 将位于其同一列或行的所有其他零元素标记为 $\emptyset$

\sectionline
\mcompress\mcompress
\section{博弈论}


行决策人:一行对应其一个选项,总选项数为行数。列决策人反之。赢得矩阵是行决策人的收益。

纯策略:

如果每行最小里的最大值 = 每列最大里的最小值:$V_{G}=\max _{i} \min _{j} a_{i j}=\min _{j} \max _{i} a_{i j}$,则 $V_{G}$ 为纯策略解。

混合策略图解法,要求其中一方的可选决策数为2,设其为 M:

\begin{enumerate}
	\item 设M取两种策略的概率分别为 $x$ 和 $1-x$
	\item 对于对手的每一种策略,算出M的收益(数学期望,用 $x$ 的式子表示)
	\item 若M是行决策者,对以上几个收益的数学期望,画图求其最小中的最大;若M是列决策者,求其最大值中的最小。
\end{enumerate}


\mcompress

\begin{figure}[h]
\centering
\begin{minipage}{0.3\linewidth}
	\includegraphics[width=\linewidth]{row}
	{\scriptsize M行决策者,B点为最优解}
\end{minipage}
\hspace{0.1\linewidth}
\begin{minipage}{0.3\linewidth}
	\includegraphics[width=\linewidth]{column}
	{\scriptsize M列决策者,BC线段最优}
\end{minipage}
\end{figure}

\mcompress
最终最优解,M的决策和$V_G$的值由代回$x$解出。M的对手N的决策判断如下:

\begin{itemize}
	\item 若如左图,最优解为顶点,则不过此顶点的直线对应的N的决策的概率为0,过此顶点的决策概率用$V_G$设未知数列方程解出
	\item 若如右图,最优解为线段,则这条线段的策略是N的最优纯策略
\end{itemize}

例如矩阵为 $
\left(
\begin{array}{ccc}
2&3&11\\
7&5&2
\end{array}
\right)
$,解的过程对应上左图,最终$V_G$为$49/11$。则解列决策人最优解的方程组为$
\left\lbrace 
\begin{array}{rll}
	3y_2 & +11y_3 & =49/11  \\
	5y_2 & + 2y_3 & = 49/11 \\
	y_2  &  +y_3  &   = 1
\end{array}
\right.
$

优超策略:如果有一行R的每个元素都\textbf{小于任意}另一行的对应元素,则R可以直接删去。如果有一列C的每个元素都大于任意\textbf{大于任意}另一列的对应元素,则C可以直接删去。

方程组法(不常用):解$
\left\lbrace
\begin{array}{rc}
\sum_i a_{ij}x_i &= v\\
\sum_i x_i &= 1
\end{array}
\right.
$, $\left\lbrace
\begin{array}{rc}
\sum_j a_{ij}y_j &= v\\
\sum_j y_j &= 1
\end{array}
\right.$两个方程组

线性规划法,直接看例子:

赢得矩阵为 $\left(
\begin{array}{ccc}
7&2&9\\
2&9&0\\
9&0&11
\end{array}
\right)
$,则两个线性规划问题为$
\left\lbrace
\begin{array}{rlll}
	\multicolumn{4}{c}{\min(x_1+x_2+x_3)} \\
	7x_1 & +2x_2  & +9x_3  & \geq 1      \\
	2x_1 & + 9x_2 &        & \geq 1      \\
	9x_1 &        & +11x_3 & \geq1
\end{array}
\right.
$ 和 $
\left\lbrace
\begin{array}{rlll}
\multicolumn{4}{c}{\max(y_1+y_2+y_3)} \\
7y_1 & +2y_2  & +9y_3  & \leq 1      \\
2y_1 & + 9y_2 &        & \leq 1      \\
9y_1 &        & +11y_3 & \leq1
\end{array}
\right.
$,注意两者互为对偶问题,只是例子中A矩阵对称因而无法体现。Max 问题照抄原矩阵,Min问题的系数为 $A^T$.

线性规划的解为 $x = (\frac{1}{20}, \frac{1}{10}, \frac{1}{20})^T, y = (\frac{1}{20}, \frac{1}{10}, \frac{1}{20})^T$ 目标函数值均为 $\frac{1}{5}$.

则决策问题的解为 $V_G = \frac{1}{\omega} = 5$, $x^* = V_Gx = (\frac{1}{4}, \frac{1}{2}, \frac{1}{4})^T, y^* = V_Gy = (\frac{1}{4}, \frac{1}{2}, \frac{1}{4})^T$






\end{document}







